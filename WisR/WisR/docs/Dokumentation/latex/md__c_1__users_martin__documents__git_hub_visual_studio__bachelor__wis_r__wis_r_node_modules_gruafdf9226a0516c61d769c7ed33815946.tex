graceful-\/fs functions as a drop-\/in replacement for the fs module, making various improvements.

The improvements are meant to normalize behavior across different platforms and environments, and to make filesystem access more resilient to errors.

\subsection*{Improvements over \href{http://api.nodejs.org/fs.html}{\tt fs module}}

graceful-\/fs\+:


\begin{DoxyItemize}
\item Queues up {\ttfamily open} and {\ttfamily readdir} calls, and retries them once something closes if there is an E\+M\+F\+I\+L\+E error from too many file descriptors.
\item fixes {\ttfamily lchmod} for Node versions prior to 0.\+6.\+2.
\item implements {\ttfamily fs.\+lutimes} if possible. Otherwise it becomes a noop.
\item ignores {\ttfamily E\+I\+N\+V\+A\+L} and {\ttfamily E\+P\+E\+R\+M} errors in {\ttfamily chown}, {\ttfamily fchown} or {\ttfamily lchown} if the user isn\textquotesingle{}t root.
\item makes {\ttfamily lchmod} and {\ttfamily lchown} become noops, if not available.
\item retries reading a file if {\ttfamily read} results in E\+A\+G\+A\+I\+N error.
\end{DoxyItemize}

On Windows, it retries renaming a file for up to one second if {\ttfamily E\+A\+C\+C\+E\+S\+S} or {\ttfamily E\+P\+E\+R\+M} error occurs, likely because antivirus software has locked the directory.

\subsection*{U\+S\+A\+G\+E}


\begin{DoxyCode}
1 // use just like fs
2 var fs = require('graceful-fs')
3 
4 // now go and do stuff with it...
5 fs.readFileSync('some-file-or-whatever')
\end{DoxyCode}
 