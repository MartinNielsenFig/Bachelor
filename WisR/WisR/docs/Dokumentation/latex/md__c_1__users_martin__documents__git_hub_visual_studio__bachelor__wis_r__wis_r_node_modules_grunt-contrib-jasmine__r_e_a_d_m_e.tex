\begin{quote}
Run jasmine specs headlessly through Phantom\+J\+S \end{quote}


\subsection*{Getting Started}

This plugin requires Grunt {\ttfamily $>$=0.\+4.\+0}

If you haven\textquotesingle{}t used \href{http://gruntjs.com/}{\tt Grunt} before, be sure to check out the \href{http://gruntjs.com/getting-started}{\tt Getting Started} guide, as it explains how to create a \href{http://gruntjs.com/sample-gruntfile}{\tt Gruntfile} as well as install and use Grunt plugins. Once you\textquotesingle{}re familiar with that process, you may install this plugin with this command\+:


\begin{DoxyCode}
1 npm install grunt-contrib-jasmine --save-dev
\end{DoxyCode}


Once the plugin has been installed, it may be enabled inside your Gruntfile with this line of Java\+Script\+:


\begin{DoxyCode}
grunt.loadNpmTasks(\textcolor{stringliteral}{'grunt-contrib-jasmine'});
\end{DoxyCode}


\subsection*{Jasmine task}

{\itshape Run this task with the {\ttfamily grunt jasmine} command.}

Automatically builds and maintains your spec runner and runs your tests headlessly through Phantom\+J\+S.

\paragraph*{Run specs locally or on a remote server}

Run your tests on your local filesystem or via a server task like \href{https://github.com/gruntjs/grunt-contrib-connect}{\tt grunt-\/contrib-\/connect}.

\paragraph*{Customize your Spec\+Runner with templates}

Use your own Spec\+Runner templates to customize how {\ttfamily grunt-\/contrib-\/jasmine} builds the Spec\+Runner. See the \href{https://github.com/gruntjs/grunt-contrib-jasmine/wiki/Jasmine-Templates}{\tt wiki} for details and third party templates for examples.

\subparagraph*{A\+M\+D Support}

Supports A\+M\+D tests via the \href{https://github.com/jsoverson/grunt-template-jasmine-requirejs}{\tt grunt-\/template-\/jasmine-\/requirejs} module

\subparagraph*{Third party templates}


\begin{DoxyItemize}
\item \href{https://github.com/jsoverson/grunt-template-jasmine-requirejs}{\tt Require\+J\+S}
\item \href{https://github.com/maenu/grunt-template-jasmine-istanbul}{\tt Code coverage output with Istanbul}
\item \href{https://github.com/jaredstehler/grunt-template-jasmine-steal}{\tt Steal\+J\+S}
\end{DoxyItemize}

\subsubsection*{Options}

\paragraph*{src}

Type\+: {\ttfamily String$\vert$\+Array}

Your source files. These are the files that you are testing. If you are using Require\+J\+S your source files will be loaded as dependencies into your spec modules and will not need to be placed here.

\paragraph*{options.\+specs}

Type\+: {\ttfamily String$\vert$\+Array}

Your Jasmine specs.

\paragraph*{options.\+vendor}

Type\+: {\ttfamily String$\vert$\+Array}

Third party libraries like j\+Query \& generally anything loaded before source, specs, and helpers.

\paragraph*{options.\+helpers}

Type\+: {\ttfamily String$\vert$\+Array}

Non-\/source, non-\/spec helper files. In the default runner these are loaded after {\ttfamily vendor} files

\paragraph*{options.\+styles}

Type\+: {\ttfamily String$\vert$\+Array}

C\+S\+S files that get loaded after the jasmine.\+css

\paragraph*{options.\+version}

Type\+: {\ttfamily String} Default\+: \textquotesingle{}2.\+0.\+1\textquotesingle{}

This is the jasmine-\/version which will be used. currently available versions are\+:


\begin{DoxyItemize}
\item 2.\+0.\+1
\item 2.\+0.\+0
\end{DoxyItemize}

{\itshape Due to changes in Jasmine, pre-\/2.\+0 versions have been dropped and tracking will resume at 2.\+0.\+0}

\paragraph*{options.\+outfile}

Type\+: {\ttfamily String} Default\+: {\ttfamily \+\_\+\+Spec\+Runner.\+html}

The auto-\/generated specfile that phantomjs will use to run your tests. Automatically deleted upon normal runs. Use the {\ttfamily \+:build} flag to generate a Spec\+Runner manually e.\+g. {\ttfamily grunt jasmine\+:my\+Task\+:build}

\paragraph*{options.\+keep\+Runner}

Type\+: {\ttfamily Boolean} Default\+: {\ttfamily false}

Prevents the auto-\/generated specfile used to run your tests from being automatically deleted.

\paragraph*{options.\+junit.\+path}

Type\+: {\ttfamily String} Default\+: undefined

Path to output J\+Unit xml

\paragraph*{options.\+junit.\+consolidate}

Type\+: {\ttfamily Boolean} Default\+: {\ttfamily false}

Consolidate the J\+Unit X\+M\+L so that there is one file per top level suite.

\paragraph*{options.\+junit.\+template}

Type\+: {\ttfamily String} Default\+: undefined

Specify a custom J\+Unit template instead of using the default {\ttfamily junit\+Template}.

\paragraph*{options.\+host}

Type\+: {\ttfamily String} Default\+: \textquotesingle{}\textquotesingle{}

The host you want Phantom\+J\+S to connect against to run your tests.

e.\+g. if using an ad hoc server from within grunt


\begin{DoxyCode}
host : \textcolor{stringliteral}{'http://127.0.0.1:8000/'}
\end{DoxyCode}


Without a {\ttfamily host}, your specs will be run from the local filesystem.

\paragraph*{options.\+template}

Type\+: {\ttfamily String} {\ttfamily Object} Default\+: undefined

Custom template used to generate your Spec Runner. Parsed as underscore templates and provided the expanded list of files needed to build a specrunner.

You can specify an object with a {\ttfamily process} method that will be called as a template function. See the \href{https://github.com/gruntjs/grunt-contrib-jasmine/wiki/Jasmine-Templates}{\tt Template A\+P\+I Documentation} for more details.

\paragraph*{options.\+template\+Options}

Type\+: {\ttfamily Object} Default\+: {\ttfamily \{\}}

Options that will be passed to your template. Used to pass settings to the template.

\paragraph*{options.\+polyfills}

Type\+: {\ttfamily String$\vert$\+Array}

Third party polyfill libraries like json2 that are loaded at the very top before anything else. es5-\/shim is loaded automatically with this library.

\paragraph*{options.\+display}

Type\+: {\ttfamily String} Default\+: {\ttfamily full}


\begin{DoxyItemize}
\item {\ttfamily full} displays the full specs tree
\item {\ttfamily short} only displays a success or failure character for each test (useful with large suites)
\item {\ttfamily none} displays nothing
\end{DoxyItemize}

\paragraph*{options.\+summary}

Type\+: {\ttfamily Boolean} Default\+: {\ttfamily false}

Display a list of all failed tests and their failure messages

\subsubsection*{Flags}

Name\+: {\ttfamily build}

Turn on this flag in order to build a Spec\+Runner html file. This is useful when troubleshooting templates, running in a browser, or as part of a watch chain e.\+g.


\begin{DoxyCode}
watch: \{
  pivotal : \{
    files: [\textcolor{stringliteral}{'src/**/*.js'}, \textcolor{stringliteral}{'specs/**/*.js'}],
    tasks: \textcolor{stringliteral}{'jasmine:pivotal:build'}
  \}
\}
\end{DoxyCode}


\subsubsection*{Filtering specs}

{\bfseries filename} {\ttfamily grunt jasmine -\/-\/filter=foo} will run spec files that have {\ttfamily foo} in their file name.

{\bfseries folder} {\ttfamily grunt jasmine -\/-\/filter=/foo} will run spec files within folders that have {\ttfamily foo$\ast$} in their name.

{\bfseries wildcard} {\ttfamily grunt jasmine -\/-\/filter=/$\ast$-\/bar} will run anything that is located in a folder {\ttfamily $\ast$-\/bar}

{\bfseries comma separated filters} {\ttfamily grunt jasmine -\/-\/filter=foo,bar} will run spec files that have {\ttfamily foo} or {\ttfamily bar} in their file name.

{\bfseries flags with space} {\ttfamily grunt jasmine -\/-\/filter=\char`\"{}foo bar\char`\"{}} will run spec files that have {\ttfamily foo bar} in their file name. {\ttfamily grunt jasmine -\/-\/filter=\char`\"{}/foo bar\char`\"{}} will run spec files within folders that have {\ttfamily foo bar$\ast$} in their name.

\paragraph*{Example application usage}


\begin{DoxyItemize}
\item \href{https://github.com/jsoverson/grunt-contrib-jasmine-example}{\tt Pivotal Labs\textquotesingle{} sample application}
\end{DoxyItemize}

\paragraph*{Basic Use}

Sample configuration to run Pivotal Labs\textquotesingle{} example Jasmine application.


\begin{DoxyCode}
\textcolor{comment}{// Example configuration}
grunt.initConfig(\{
  jasmine: \{
    pivotal: \{
      src: \textcolor{stringliteral}{'src/**/*.js'},
      options: \{
        specs: \textcolor{stringliteral}{'spec/*Spec.js'},
        helpers: \textcolor{stringliteral}{'spec/*Helper.js'}
      \}
    \}
  \}
\});
\end{DoxyCode}


\paragraph*{Supplying a custom template}

Supplying a custom template to the above example


\begin{DoxyCode}
\textcolor{comment}{// Example configuration}
grunt.initConfig(\{
  jasmine: \{
    customTemplate: \{
      src: \textcolor{stringliteral}{'src/**/*.js'},
      options: \{
        specs: \textcolor{stringliteral}{'spec/*Spec.js'},
        helpers: \textcolor{stringliteral}{'spec/*Helper.js'},
        \textcolor{keyword}{template}: \textcolor{stringliteral}{'custom.tmpl'}
      \}
    \}
  \}
\});
\end{DoxyCode}


\paragraph*{Supplying template modules and vendors}

A complex version for the above example


\begin{DoxyCode}
\textcolor{comment}{// Example configuration}
grunt.initConfig(\{
  jasmine: \{
    customTemplate: \{
      src: \textcolor{stringliteral}{'src/**/*.js'},
      options: \{
        specs: \textcolor{stringliteral}{'spec/*Spec.js'},
        helpers: \textcolor{stringliteral}{'spec/*Helper.js'},
        \textcolor{keyword}{template}: require(\textcolor{stringliteral}{'exports-process.js'})
        vendor: [
          "vendor\textcolor{comment}{/*.js",}
\textcolor{comment}{          "http://ajax.googleapis.com/ajax/libs/jquery/1.11.0/jquery.min.js"}
\textcolor{comment}{        ]}
\textcolor{comment}{      \}}
\textcolor{comment}{    \}}
\textcolor{comment}{  \}}
\textcolor{comment}{\});}
\end{DoxyCode}


\paragraph*{Sample Require\+J\+S/\+N\+P\+M Template usage}


\begin{DoxyCode}
\textcolor{comment}{// Example configuration}
grunt.initConfig(\{
  jasmine: \{
    yourTask: \{
      src: \textcolor{stringliteral}{'src/**/*.js'},
      options: \{
        specs: \textcolor{stringliteral}{'spec/*Spec.js'},
        \textcolor{keyword}{template}: require(\textcolor{stringliteral}{'grunt-template-jasmine-requirejs'})
      \}
    \}
  \}
\});
\end{DoxyCode}


N\+P\+M Templates are just node modules, so you can write and treat them as such.

Please see the \href{https://github.com/jsoverson/grunt-template-jasmine-requirejs}{\tt grunt-\/template-\/jasmine-\/requirejs} documentation for more information on the Require\+J\+S template.

\subsection*{Release History}


\begin{DoxyItemize}
\item 2015-\/09-\/24   v0.9.\+2   \+Fixes npm@3 issues
\item 2015-\/09-\/04   v0.9.\+1   \+Fix summary logging
\item 2015-\/07-\/10   v0.9.\+0   \+Fix deprecated package.\+json licenses. Fix Phantomjs dependency to include correct phantom kill
\item 2015-\/01-\/08   v0.8.\+2   \+Fixes to test summary reporting.
\item 2014-\/10-\/20   v0.8.\+1   \+Now removes listeners when using the build flag. Adds handler for fail.\+load.
\item 2014-\/07-\/26   v0.8.\+0   \+Plugin now uses Jasmine 2.\+0.\+4 from npm. Updates other dependencies. Added \char`\"{}options.\+polyfills\char`\"{}.
\item 2014-\/07-\/26   v0.7.\+0   \+Merged 153 to add stack trace to summary. Updated for Jasmine 2.\+0.\+1 Merged 133 for minimal output Merged 139 changing file exclusion logic
\item 2014-\/05-\/31   v0.6.\+5   \+Option to allow specifying a junit\+Template.
\item 2014-\/04-\/28   v0.6.\+4   \+Indent level fix. Moved scripts inside the body tag.
\item 2014-\/01-\/29   v0.6.\+0   \+Jasmine 2.\+0.\+0 support Improved logging support Various merges/bugfixes
\item 2013-\/08-\/02   v0.5.\+2   \+Fixed breakage with iframes /44 Added filter flag / 70 Fixed junit failure output /77
\item 2013-\/06-\/18   v0.5.\+1   \+Merged /69 grunt async not called when tests fail O\+R keep\+Runner is true
\item 2013-\/06-\/15   v0.5.\+0   updated rimraf made teardown async, added Function.\+prototype.\+bind polyfill breaking (templates) changed input options for get\+Relative\+File\+List breaking (usage) failing task on phantom error (Syntax\+Error, Type\+Error, et al)
\item 2013-\/04-\/03   v0.4.\+2   bumped grunt-\/lib-\/phantomjs to 0.\+3.\+0/1.9 (closes merged addressed
\item 2013-\/03-\/08   v0.4.\+0   bumped grunt-\/lib-\/phantomjs to 0.\+2.\+0/1.8 allowed spec/vendor/helper list to return non-\/matching files (e.\+g. for remote, http) merged merged
\item 2013-\/02-\/24   v0.3.\+3   \+Added better console output (via Gabor Kiss )
\item 2013-\/02-\/17   v0.3.\+2   \+Ensure Gruntfile.\+js is included on npm.
\item 2013-\/02-\/15   v0.3.\+1   \+First official release for Grunt 0.\+4.\+0.
\item 2013-\/01-\/22   v0.3.\+1rc7   \+Exposed phantom and send\+Message to templates
\item 2013-\/01-\/22   v0.3.\+0rc7   \+Updated dependencies for grunt v0.\+4.\+0rc6/rc7
\item 2013-\/01-\/08   v0.3.\+0rc5   \+Updating to work with grunt v0.\+4.\+0rc5. Switching to this.\+files\+Src api. Added J\+Unit xml output (via Kelvin Luck ) Passing console.\+log from browser to verbose grunt logging Support for templates as separate node modules Removed internal requirejs template (see grunt-\/template-\/jasmine-\/requirejs)
\item 2012-\/12-\/03   v0.2.\+0   \+Generalized requirejs template config Added loader plugin Tests for templates Updated jasmine to 1.\+3.\+0
\item 2012-\/11-\/24   v0.1.\+2   \+Updated for new grunt/grunt-\/contrib apis
\item 2012-\/11-\/07   v0.1.\+1   \+Fixed race condition in requirejs template
\item 2012-\/11-\/07   v0.1.\+0   \+Ported grunt-\/jasmine-\/runner and grunt-\/jasmine-\/task to grunt-\/contrib 


\end{DoxyItemize}

Task submitted by \href{http://jarrodoverson.com}{\tt Jarrod Overson}

{\itshape This file was generated on Thu Sep 24 2015 13\+:00\+:10.} 