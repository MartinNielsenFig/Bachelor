\subsection*{Reporting Issues}

Please see our https\+://github.com/wycats/handlebars.\+js/blob/master/\+F\+A\+Q.md \char`\"{}\+F\+A\+Q\char`\"{} for common issues that people run into.

Should you run into other issues with the project, please don\textquotesingle{}t hesitate to let us know by filing an \href{https://github.com/wycats/handlebars.js/issues/new}{\tt issue}! In general we are going to ask for an example of the problem failing, which can be as simple as a jsfiddle/jsbin/etc. We\textquotesingle{}ve put together a jsfiddle \href{http://jsfiddle.net/9D88g/46/}{\tt template} to ease this. (We will keep this link up to date as new releases occur, so feel free to check back here)

Pull requests containing only failing thats demonstrating the issue are welcomed and this also helps ensure that your issue won\textquotesingle{}t regress in the future once it\textquotesingle{}s fixed.

Documentation issues on the handlebarsjs.\+com site should be reported on \href{https://github.com/wycats/handlebars-site}{\tt handlebars-\/site}.

\subsection*{Pull Requests}

We also accept \href{https://github.com/wycats/handlebars.js/pull/new/master}{\tt pull requests}!

Generally we like to see pull requests that
\begin{DoxyItemize}
\item Maintain the existing code style
\item Are focused on a single change (i.\+e. avoid large refactoring or style adjustments in untouched code if not the primary goal of the pull request)
\item Have \href{http://tbaggery.com/2008/04/19/a-note-about-git-commit-messages.html}{\tt good commit messages}
\item Have tests
\item Don\textquotesingle{}t significantly decrease the current code coverage (see coverage/lcov-\/report/index.\+html)
\end{DoxyItemize}

\subsection*{Building}

To build Handlebars.\+js you\textquotesingle{}ll need a few things installed.


\begin{DoxyItemize}
\item Node.\+js
\item \href{http://gruntjs.com/getting-started}{\tt Grunt}
\end{DoxyItemize}

Before building, you need to make sure that the Git submodule {\ttfamily spec/mustache} is included (i.\+e. the directory {\ttfamily spec/mustache} should not be empty). To include it, if using Git version 1.\+6.\+5 or newer, use {\ttfamily git clone -\/-\/recursive} rather than {\ttfamily git clone}. Or, if you already cloned without {\ttfamily -\/-\/recursive}, use {\ttfamily git submodule update -\/-\/init}.

Project dependencies may be installed via {\ttfamily npm install}.

To build Handlebars.\+js from scratch, you\textquotesingle{}ll want to run {\ttfamily grunt} in the root of the project. That will build Handlebars and output the results to the dist/ folder. To re-\/run tests, run {\ttfamily grunt test} or {\ttfamily npm test}. You can also run our set of benchmarks with {\ttfamily grunt bench}.

The {\ttfamily grunt dev} implements watching for tests and allows for in browser testing at {\ttfamily \href{http://localhost:9999/spec/}{\tt http\+://localhost\+:9999/spec/}}.

If you notice any problems, please report them to the Git\+Hub issue tracker at \href{http://github.com/wycats/handlebars.js/issues}{\tt http\+://github.\+com/wycats/handlebars.\+js/issues}.

\subsection*{Ember testing}

The current ember distribution should be tested as part of the handlebars release process. This requires building the {\ttfamily handlebars-\/source} gem locally and then executing the ember test script.


\begin{DoxyCode}
1 npm link
2 grunt build release
3 cp dist/*.js $emberRepoDir/bower\_components/handlebars/
4 
5 cd $emberRepoDir
6 npm link handlebars
7 npm test
\end{DoxyCode}


\subsection*{Releasing}

Handlebars utilizes the \href{https://github.com/walmartlabs/generator-release}{\tt release yeoman generator} to perform most release tasks.

A full release may be completed with the following\+:


\begin{DoxyCode}
1 yo release
2 npm publish
3 yo release:publish components handlebars.js dist/components/
4 
5 cd dist/components/
6 gem build handlebars-source.gemspec
7 gem push handlebars-source-*.gem
\end{DoxyCode}


After this point the handlebars site needs to be updated to point to the new version numbers. The jsfiddle link should be updated to point to the most recent distribution for all instances in our documentation. 