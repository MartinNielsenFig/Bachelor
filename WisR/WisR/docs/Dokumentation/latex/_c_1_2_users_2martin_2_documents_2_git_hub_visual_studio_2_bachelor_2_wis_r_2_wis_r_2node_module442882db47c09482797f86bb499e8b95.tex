\hypertarget{_c_1_2_users_2martin_2_documents_2_git_hub_visual_studio_2_bachelor_2_wis_r_2_wis_r_2node_module442882db47c09482797f86bb499e8b95}{}\section{C\+:/\+Users/martin/\+Documents/\+Git\+Hub\+Visual\+Studio/\+Bachelor/\+Wis\+R/\+Wis\+R/node\+\_\+modules/grunt-\/ngdocs/\+R\+E\+A\+D\+M\+E.\+md}
$<$file name=\char`\"{}index.\+html\char`\"{}$>$ $<$textarea ng-\/model=\char`\"{}text\char`\"{}rx-\/autogrow class=\char`\"{}input-\/block-\/level\char`\"{}$>$$<$/textarea$>$ 
\begin{DoxyPre}\{\{text\}\}\end{DoxyPre}
 $<$/file$>$ 

$\ast$/ angular.\+module(\textquotesingle{}rfx\textquotesingle{}, \mbox{[}\mbox{]}).directive(\textquotesingle{}r\+Autogrow\textquotesingle{}, function() \{ //some nice code \}); ```

Check out the \href{https://github.com/angular/angular.js/wiki/Writing-AngularJS-Documentation}{\tt Writing Angular\+J\+S documentation wiki article} to see what\textquotesingle{}s possible, or take a look at the \href{https://github.com/angular/angular.js/tree/master/src/ng}{\tt Angular\+J\+S source code} for more examples.

\subsubsection*{Batarang}

If your examples are empty you maybe have batarang enabled for the docs site. This is the same issue as on \href{http://docs.angular.js}{\tt http\+://docs.\+angular.\+js} and the batarang team is informed about it \#68.

\subsubsection*{License}

M\+I\+T License


\begin{DoxyCodeInclude}
1 # grunt-ngdocs
2 Grunt plugin to create a documentation like [AngularJS](http://docs.angularjs.org)
3 NOTE: this plugin requires Grunt 0.4.x
4 
5 ATTENTION: grunt-ngdocs 0.2+ is for angularjs 1.2+
6 grunt-ngdocs 0.2.5 supports angularjs 1.3+ too
7 Please include angular.js and angular-animate.js with the scripts option
8 
9 ## Getting Started
10 From the same directory as your project's Gruntfile and package.json, install this plugin with the
       following command:
11 
12 `npm install grunt-ngdocs --save-dev`
13 
14 Once that's done, add this line to your project's Gruntfile:
15 
16 ```js
17 grunt.loadNpmTasks('grunt-ngdocs');
18 ```
19 
20 A full working example can be found at
       [https://github.com/m7r/grunt-ngdocs-example](https://github.com/m7r/grunt-ngdocs-example)
21 
22 ## Config
23 Inside your `Gruntfile.js` file, add a section named *ngdocs*.
24 Here's a simple example:
25 
26 ```js
27 ngdocs: \{
28   all: ['src/**/*.js']
29 \}
30 ```
31 
32 And with many options:
33 
34 ```js
35 ngdocs: \{
36   options: \{
37     dest: 'docs',
38     scripts: ['../app.min.js'],
39     html5Mode: true,
40     startPage: '/api',
41     title: "My Awesome Docs",
42     image: "path/to/my/image.png",
43     imageLink: "http://my-domain.com",
44     titleLink: "/api",
45     inlinePartials: true,
46     bestMatch: true,
47     analytics: \{
48           account: 'UA-08150815-0'
49     \},
50     discussions: \{
51           shortName: 'my',
52           url: 'http://my-domain.com',
53           dev: false
54     \}
55   \},
56   tutorial: \{
57     src: ['content/tutorial/*.ngdoc'],
58     title: 'Tutorial'
59   \},
60   api: \{
61     src: ['src/**/*.js', '!src/**/*.spec.js'],
62     title: 'API Documentation'
63   \}
64 \}
65 ```
66 
67 
68 ### Targets
69 Each grunt target creates a section in the documentation app.
70 
71 #### src
72 [required] List of files to parse for documentation comments.
73 
74 #### title
75 [default] 'API Documentation'
76 
77 Set the name for the section in the documentation app.
78 
79 #### api
80 [default] true for target api
81 
82 Set the sidebar to advanced mode, with sections for modules, services, etc.
83 
84 
85 ### Options
86 
87 #### dest
88 [default] 'docs'
89 
90 Folder relative to your Gruntfile where the documentation should be built.
91 
92 #### scripts
93 [default] ['angular.js']
94 
95 Set which angular.js file or addional custom js files are loaded to the app. This allows the live examples
       to use custom directives, services, etc. The documentation app works with angular.js 1.2+ and 1.3+. If you
       include your own angular.js include angular-animate.js too.
96 
97 Possible values:
98 
99   - ['angular.js'] use angular and angular-animate 1.2.16 delivered with grunt-ngdocs
100   - ['path/to/file.js'] file will be copied into the docs, into a `grunt-scripts` folder
101   - ['http://example.com/file.js', 'https://example.com/file.js', '//example.com/file.js'] reference remote
       files (eg from a CDN)
102   - ['../app.js'] reference file relative to the dest folder
103 
104 #### deferLoad
105 [default] false
106 
107 If you want to use requirejs as loader set this to `true`.
108 
109 Include 'js/angular-bootstrap.js', 'js/angular-bootstrap-prettify.js', 'js/docs-setup.js', 'js/docs.js'
       with requirejs and finally bootstrap the app `angular.bootstrap(document, ['docsApp']);`.
110 
111 #### styles
112 [default] []
113 
114 Copy additional css files to the documentation app
115 
116 #### template
117 [default] null
118 
119 Allow to use your own template. Use the default template at src/templates/index.tmpl as reference.
120 
121 #### startPage
122 [default] '/api'
123 
124 Set first page to open.
125 
126 #### html5Mode
127 [default] false
128 
129 Whether or not to enable `html5Mode` in the docs application.  If true, then links will be absolute.  If
       false, they will be prefixed by `#/`.
130 
131 #### bestMatch
132 [default] false
133 
134 The best matching page for a search query is highlighted and get selected on return.
135 If this option is set to true the best match is shown below the search field in an dropdown menu. Use this
       for long lists where the highlight is often not visible.
136 
137 
138 #### title
139 [default] "name" or "title" field in `pkg`
140 
141 Title to put on the navbar and the page's `title` attribute.  By default, tries to
142 find the title in the `pkg`. If it can't find it, it will go to an empty string.
143 
144 #### titleLink
145 [default] no anchor tag is used
146 
147 Wraps the title text in an anchor tag with the provided URL.
148 
149 #### image
150 A URL or relative path to an image file to use in the top navbar.
151 
152 #### imageLink
153 [default] no anchor tag is used
154 
155 Wraps the navbar image in an anchor tag with the provided URL.
156 
157 #### navTemplate
158 [default] null
159 
160 Path to a template of a nav HTML template to include.  The css for it
161 should be that of listitems inside a bootstrap navbar:
162 
163 ```html
164 <header class="header">
165   <div class="navbar">
166     <ul class="nav">
167       \{\{links to all the docs pages\}\}
168     </ul>
169     \{\{YOUR\_NAV\_TEMPLATE\_GOES\_HERE\}\}
170   </div>
171 </header>
172 ```
173 Example: 'templates/my-nav.html'
174 
175 The template, if specified, is pre-processed using
       [grunt.template](https://github.com/gruntjs/grunt/wiki/grunt.template#grunttemplateprocess).
176 
177 
178 #### sourceLink
179 [default] true
180 
181 Display "View source" link.
182 Possible values are
183 
184   - `true`: try to read repository from package.json (currently only github is supported)
185   - `false`: don't display link
186   - string: template string like `'https://internal.server/repo/blob/\{\{sha\}\}/\{\{file\}\}#L\{\{codeline\}\}'`
187 
188     available placeholders:
189 
190       - **file**: path and filename current file
191       - **filename**: only filename of current file
192       - **filepath**: directory of current file
193       - **line**: first line of comment
194       - **codeline**: first line *after* comment
195       - **version**: version read from package.json
196       - **sha**: first 7 characters of current git commit
197 
198 #### editLink
199 [default] true
200 
201 Display "Improve this doc" link. Same options as for sourceLink.
202 
203 #### editExample
204 [default] true
205 
206 Show Edit Button for examples.
207 
208 #### inlinePartials
209 [default] false
210 
211 If set to true this option will turn all partials into angular inline templates and place them inside the
       generated `index.html` file.
212 The advantage over lazyloading with ajax is that the documentation will also work on the `file://` system.
213 
214 #### discussions
215 Optional include [discussions](http://disqus.com) in the documentation app.
216 
217 ```js
218 \{
219   shortName: 'my',
220   url: 'http://my-domain.com',
221   dev: false
222 \}
223 ```
224 
225 #### analytics
226 Optional include Google Analytics in the documentation app.
227 
228 ```js
229 \{
230   account: 'UA-08150815-0'
231 \}
232 ```
233 
234 
235 ## How it works
236 The task parses the specified files for doc comments and extracts them into partial html files for the
       documentation app.
237 
238 At first run, all necessary files will be copied to the destination folder.
239 After that, only index.html, js/docs-setup.js, and the partials will be overwritten.
240 
241 Partials that are no longer needed will not be deleted. Use, for example, the grunt-contrib-clean task to
       clean the docs folder before creating a distribution build.
242 
243 After an update of grunt-ngdocs you should clean the docs folder too.
244 
245 A doc comment looks like this:
246 
247 ```js
248 /**
249  * @ngdoc directive
250  * @name rfx.directive:rAutogrow
251  * @element textarea
252  * @function
253  *
254  * @description
255  * Resize textarea automatically to the size of its text content.
256  *
257  * **Note:** ie<9 needs polyfill for window.getComputedStyle
258  *
259  * @example
260    <example module="rfx">
261      <file name="index.html">
262          <textarea ng-model="text"rx-autogrow class="input-block-level"></textarea>
263          <pre>\{\{text\}\}</pre>
264      </file>
265    </example>
266  */
267 angular.module('rfx', []).directive('rAutogrow', function() \{
268   //some nice code
269 \});
270 ```
271 
272 Check out the [Writing AngularJS documentation wiki
       article](https://github.com/angular/angular.js/wiki/Writing-AngularJS-Documentation) to see what's possible,
273 or take a look at the [AngularJS source code](https://github.com/angular/angular.js/tree/master/src/ng) for
       more examples.
274 
275 ## Batarang
276 If your examples are empty you maybe have batarang enabled for the docs site.
277 This is the same issue as on http://docs.angular.js and the batarang team is informed about it #68.
278 
279 ## License
280 MIT License
\end{DoxyCodeInclude}
 