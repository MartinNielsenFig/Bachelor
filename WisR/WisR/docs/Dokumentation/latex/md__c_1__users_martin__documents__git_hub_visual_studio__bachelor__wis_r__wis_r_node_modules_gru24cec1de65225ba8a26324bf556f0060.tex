\href{http://travis-ci.org/jprichardson/node-fs-extra}{\tt !\mbox{[}build status\mbox{]}(https\+://secure.\+travis-\/ci.\+org/jprichardson/node-\/fs-\/extra.\+svg)} \href{https://ci.appveyor.com/project/jprichardson/node-fs-extra/branch/master}{\tt !\mbox{[}windows Build status\mbox{]}(https\+://img.\+shields.\+io/appveyor/ci/jprichardson/node-\/fs-\/extra/master.\+svg?label=windows\%20build)} \href{https://www.npmjs.org/package/fs-extra}{\tt !\mbox{[}downloads per month\mbox{]}(http\+://img.\+shields.\+io/npm/dm/fs-\/extra.\+svg)} \href{https://coveralls.io/r/jprichardson/node-fs-extra}{\tt !\mbox{[}Coverage Status\mbox{]}(https\+://img.\+shields.\+io/coveralls/jprichardson/node-\/fs-\/extra.\+svg)}

{\ttfamily fs-\/extra} adds file system methods that aren\textquotesingle{}t included in the native {\ttfamily fs} module. It is a drop in replacement for {\ttfamily fs}.

\subsection*{Why? }

I got tired of including {\ttfamily mkdirp}, {\ttfamily rimraf}, and {\ttfamily cp -\/r} in most of my projects.

\subsection*{Installation }

\begin{DoxyVerb}npm install --save fs-extra
\end{DoxyVerb}


\subsection*{Usage }

{\ttfamily fs-\/extra} is a drop in replacement for native {\ttfamily fs}. All methods in {\ttfamily fs} are unmodified and attached to {\ttfamily fs-\/extra}.

You don\textquotesingle{}t ever need to include the original {\ttfamily fs} module again\+:


\begin{DoxyCode}
var fs = require(\textcolor{stringliteral}{'fs'}) \textcolor{comment}{// this is no longer necessary}
\end{DoxyCode}


you can now do this\+:


\begin{DoxyCode}
var fs = require(\textcolor{stringliteral}{'fs-extra'})
\end{DoxyCode}


or if you prefer to make it clear that you\textquotesingle{}re using {\ttfamily fs-\/extra} and not {\ttfamily fs}, you may want to name your {\ttfamily fs} variable {\ttfamily fse} like so\+:


\begin{DoxyCode}
var fse = require(\textcolor{stringliteral}{'fs-extra'})
\end{DoxyCode}


you can also keep both, but it\textquotesingle{}s redundant\+:


\begin{DoxyCode}
var fs = require(\textcolor{stringliteral}{'fs'})
var fse = require('fs-extra')
\end{DoxyCode}


\subsection*{Methods }


\begin{DoxyItemize}
\item \href{#copy}{\tt copy}
\item \href{#copy}{\tt copy\+Sync}
\item \href{#createoutputstreamfile-options}{\tt create\+Output\+Stream}
\item \href{#emptydirdir-callback}{\tt empty\+Dir}
\item \href{#emptydirdir-callback}{\tt empty\+Dir\+Sync}
\item \href{#ensurefilefile-callback}{\tt ensure\+File}
\item \href{#ensurefilefile-callback}{\tt ensure\+File\+Sync}
\item \href{#ensuredirdir-callback}{\tt ensure\+Dir}
\item \href{#ensuredirdir-callback}{\tt ensure\+Dir\+Sync}
\item \href{#ensurelinksrcpath-dstpath-callback}{\tt ensure\+Link}
\item \href{#ensurelinksrcpath-dstpath-callback}{\tt ensure\+Link\+Sync}
\item \href{#ensuresymlinksrcpath-dstpath-type-callback}{\tt ensure\+Symlink}
\item \href{#ensuresymlinksrcpath-dstpath-type-callback}{\tt ensure\+Symlink\+Sync}
\item \href{#mkdirsdir-callback}{\tt mkdirs}
\item \href{#mkdirsdir-callback}{\tt mkdirs\+Sync}
\item \href{#movesrc-dest-options-callback}{\tt move}
\item \href{#outputfilefile-data-callback}{\tt output\+File}
\item \href{#outputfilefile-data-callback}{\tt output\+File\+Sync}
\item \href{#outputjsonfile-data-callback}{\tt output\+Json}
\item \href{#outputjsonfile-data-callback}{\tt output\+Json\+Sync}
\item \href{#readjsonfile-options-callback}{\tt read\+Json}
\item \href{#readjsonfile-options-callback}{\tt read\+Json\+Sync}
\item \href{#removedir-callback}{\tt remove}
\item \href{#removedir-callback}{\tt remove\+Sync}
\item \href{#writejsonfile-object-options-callback}{\tt write\+Json}
\item \href{#writejsonfile-object-options-callback}{\tt write\+Json\+Sync}
\end{DoxyItemize}

{\bfseries N\+O\+T\+E\+:} You can still use the native Node.\+js methods. They are copied over to {\ttfamily fs-\/extra}.

\subsubsection*{copy()}

{\bfseries copy(src, dest, \mbox{[}options\mbox{]}, callback)}

Copy a file or directory. The directory can have contents. Like {\ttfamily cp -\/r}.

Options\+: clobber (boolean)\+: overwrite existing file or directory preserve\+Timestamps (boolean)\+: will set last modification and access times to the ones of the original source files, default is {\ttfamily false}.

Sync\+: {\ttfamily copy\+Sync()}

Examples\+:


\begin{DoxyCode}
var fs = require(\textcolor{stringliteral}{'fs-extra'})

fs.copy('/tmp/myfile', '/tmp/mynewfile', function (err) \{
  \textcolor{keywordflow}{if} (err) \textcolor{keywordflow}{return} console.error(err)
  console.log(\textcolor{stringliteral}{"success!"})
\}) \textcolor{comment}{// copies file}

fs.copy(\textcolor{stringliteral}{'/tmp/mydir'}, \textcolor{stringliteral}{'/tmp/mynewdir'}, function (err) \{
  \textcolor{keywordflow}{if} (err) \textcolor{keywordflow}{return} console.error(err)
  console.log(\textcolor{stringliteral}{'success!'})
\}) \textcolor{comment}{// copies directory, even if it has subdirectories or files}
\end{DoxyCode}


\subsubsection*{create\+Output\+Stream(file, \mbox{[}options\mbox{]})}

Exactly like {\ttfamily create\+Write\+Stream}, but if the directory does not exist, it\textquotesingle{}s created.

Examples\+:


\begin{DoxyCode}
var fs = require(\textcolor{stringliteral}{'fs-extra'})

\textcolor{comment}{// if /tmp/some does not exist, it is created}
var ws = fs.createOutputStream('/tmp/some/file.txt')
ws.write('hello\(\backslash\)n')
\end{DoxyCode}


Note on naming\+: you\textquotesingle{}ll notice that fs-\/extra has some methods like {\ttfamily fs.\+output\+Json}, {\ttfamily fs.\+output\+File}, etc that use the word {\ttfamily output} to denote that if the containing directory does not exist, it should be created. If you can think of a better succinct nomenclature for these methods, please open an issue for discussion. Thanks.

\subsubsection*{empty\+Dir(dir, \mbox{[}callback\mbox{]})}

Ensures that a directory is empty. If the directory does not exist, it is created. The directory itself is not deleted.

Alias\+: {\ttfamily emptydir()}

Sync\+: {\ttfamily empty\+Dir\+Sync()}, {\ttfamily emptydir\+Sync()}

Example\+:


\begin{DoxyCode}
var fs = require(\textcolor{stringliteral}{'fs-extra'})

\textcolor{comment}{// assume this directory has a lot of files and folders}
fs.emptyDir('/tmp/some/dir', function (err) \{
  \textcolor{keywordflow}{if} (!err) console.log(\textcolor{stringliteral}{'success!'})
\})
\end{DoxyCode}


\subsubsection*{ensure\+File(file, callback)}

Ensures that the file exists. If the file that is requested to be created is in directories that do not exist, these directories are created. If the file already exists, it is {\bfseries N\+O\+T M\+O\+D\+I\+F\+I\+E\+D}.

Alias\+: {\ttfamily create\+File()}

Sync\+: {\ttfamily create\+File\+Sync()},{\ttfamily ensure\+File\+Sync()}

Example\+:


\begin{DoxyCode}
var fs = require(\textcolor{stringliteral}{'fs-extra'})

var file = '/tmp/this/path/does/not/exist/file.txt'
fs.ensureFile(file, function (err) \{
  console.log(err) \textcolor{comment}{// => null}
  \textcolor{comment}{// file has now been created, including the directory it is to be placed in}
\})
\end{DoxyCode}


\subsubsection*{ensure\+Dir(dir, callback)}

Ensures that the directory exists. If the directory structure does not exist, it is created.

Sync\+: {\ttfamily ensure\+Dir\+Sync()}

Example\+:


\begin{DoxyCode}
var fs = require(\textcolor{stringliteral}{'fs-extra'})

var dir = '/tmp/this/path/does/not/exist'
fs.ensureDir(dir, function (err) \{
  console.log(err) \textcolor{comment}{// => null}
  \textcolor{comment}{// dir has now been created, including the directory it is to be placed in}
\})
\end{DoxyCode}


\subsubsection*{ensure\+Link(srcpath, dstpath, callback)}

Ensures that the link exists. If the directory structure does not exist, it is created.

Sync\+: {\ttfamily ensure\+Link\+Sync()}

Example\+:


\begin{DoxyCode}
var fs = require(\textcolor{stringliteral}{'fs-extra'})

var srcpath = '/tmp/file.txt'
var dstpath = '/tmp/this/path/does/not/exist/file.txt'
fs.ensureLink(srcpath, dstpath, function (err) \{
  console.log(err) \textcolor{comment}{// => null}
  \textcolor{comment}{// link has now been created, including the directory it is to be placed in}
\})
\end{DoxyCode}


\subsubsection*{ensure\+Symlink(srcpath, dstpath, \mbox{[}type\mbox{]}, callback)}

Ensures that the symlink exists. If the directory structure does not exist, it is created.

Sync\+: {\ttfamily ensure\+Symlink\+Sync()}

Example\+:


\begin{DoxyCode}
var fs = require(\textcolor{stringliteral}{'fs-extra'})

var srcpath = '/tmp/file.txt'
var dstpath = '/tmp/this/path/does/not/exist/file.txt'
fs.ensureSymlink(srcpath, dstpath, function (err) \{
  console.log(err) \textcolor{comment}{// => null}
  \textcolor{comment}{// symlink has now been created, including the directory it is to be placed in}
\})
\end{DoxyCode}


\subsubsection*{mkdirs(dir, callback)}

Creates a directory. If the parent hierarchy doesn\textquotesingle{}t exist, it\textquotesingle{}s created. Like {\ttfamily mkdir -\/p}.

Alias\+: {\ttfamily mkdirp()}

Sync\+: {\ttfamily mkdirs\+Sync()} / {\ttfamily mkdirp\+Sync()}

Examples\+:


\begin{DoxyCode}
var fs = require(\textcolor{stringliteral}{'fs-extra'})

fs.mkdirs('/tmp/some/\textcolor{keywordtype}{long}/path/that/prob/doesnt/exist', function (err) \{
  \textcolor{keywordflow}{if} (err) \textcolor{keywordflow}{return} console.error(err)
  console.log(\textcolor{stringliteral}{"success!"})
\})

fs.mkdirsSync(\textcolor{stringliteral}{'/tmp/another/path'})
\end{DoxyCode}


\subsubsection*{move(src, dest, \mbox{[}options\mbox{]}, callback)}

Moves a file or directory, even across devices.

Options\+: clobber (boolean)\+: overwrite existing file or directory limit (number)\+: number of concurrent moves, see ncp for more information

Example\+:


\begin{DoxyCode}
var fs = require(\textcolor{stringliteral}{'fs-extra'})

fs.move('/tmp/somefile', '/tmp/does/not/exist/yet/somefile', function (err) \{
  \textcolor{keywordflow}{if} (err) \textcolor{keywordflow}{return} console.error(err)
  console.log(\textcolor{stringliteral}{"success!"})
\})
\end{DoxyCode}


\subsubsection*{output\+File(file, data, callback)}

Almost the same as {\ttfamily write\+File} (i.\+e. it \href{http://pages.citebite.com/v2o5n8l2f5reb}{\tt overwrites}), except that if the parent directory does not exist, it\textquotesingle{}s created.

Sync\+: {\ttfamily output\+File\+Sync()}

Example\+:


\begin{DoxyCode}
var fs = require(\textcolor{stringliteral}{'fs-extra'})
var file = '/tmp/this/path/does/not/exist/file.txt'

fs.outputFile(file, 'hello!', function (err) \{
  console.log(err) \textcolor{comment}{// => null}

  fs.readFile(file, \textcolor{stringliteral}{'utf8'}, \textcolor{keyword}{function} (err, data) \{
    console.log(data) \textcolor{comment}{// => hello!}
  \})
\})
\end{DoxyCode}


\subsubsection*{output\+Json(file, data, \mbox{[}options\mbox{]}, callback)}

Almost the same as {\ttfamily write\+Json}, except that if the directory does not exist, it\textquotesingle{}s created. {\ttfamily options} are what you\textquotesingle{}d pass to \href{https://github.com/jprichardson/node-jsonfile#writefilefilename-options-callback}{\tt `json\+File.write\+File()`}.

Alias\+: {\ttfamily output\+J\+S\+O\+N()}

Sync\+: {\ttfamily output\+Json\+Sync()}, {\ttfamily output\+J\+S\+O\+N\+Sync()}

Example\+:


\begin{DoxyCode}
var fs = require(\textcolor{stringliteral}{'fs-extra'})
var file = '/tmp/this/path/does/not/exist/file.txt'

fs.outputJson(file, \{name: \textcolor{stringliteral}{'JP'}\}, \textcolor{keyword}{function} (err) \{
  console.log(err) \textcolor{comment}{// => null}

  fs.readJson(file, \textcolor{keyword}{function}(err, data) \{
    console.log(data.name) \textcolor{comment}{// => JP}
  \})
\})
\end{DoxyCode}


\subsubsection*{read\+Json(file, \mbox{[}options\mbox{]}, callback)}

Reads a J\+S\+O\+N file and then parses it into an object. {\ttfamily options} are the same that you\textquotesingle{}d pass to \href{https://github.com/jprichardson/node-jsonfile#readfilefilename-options-callback}{\tt `json\+File.read\+File`}.

Alias\+: {\ttfamily read\+J\+S\+O\+N()}

Sync\+: {\ttfamily read\+Json\+Sync()}, {\ttfamily read\+J\+S\+O\+N\+Sync()}

Example\+:


\begin{DoxyCode}
var fs = require(\textcolor{stringliteral}{'fs-extra'})

fs.readJson('./package.json', function (err, packageObj) \{
  console.log(packageObj.version) \textcolor{comment}{// => 0.1.3}
\})
\end{DoxyCode}


{\ttfamily read\+Json\+Sync()} can take a {\ttfamily throws} option set to {\ttfamily false} and it won\textquotesingle{}t throw if the J\+S\+O\+N is invalid. Example\+:


\begin{DoxyCode}
var fs = require(\textcolor{stringliteral}{'fs-extra'})
var file = path.join('/tmp/some-invalid.json')
var data = '\{not valid JSON\textcolor{stringliteral}{'}
\textcolor{stringliteral}{fs.writeFileSync(file, data)}
\textcolor{stringliteral}{}
\textcolor{stringliteral}{var obj = fs.readJsonSync(file, \{throws: false\})}
\textcolor{stringliteral}{console.log(obj) // => null}
\end{DoxyCode}


\subsubsection*{remove(dir, callback)}

Removes a file or directory. The directory can have contents. Like {\ttfamily rm -\/rf}.

Alias\+: {\ttfamily delete()}

Sync\+: {\ttfamily remove\+Sync()} / {\ttfamily delete\+Sync()}

Examples\+:


\begin{DoxyCode}
var fs = require(\textcolor{stringliteral}{'fs-extra'})

fs.remove('/tmp/myfile', function (err) \{
  \textcolor{keywordflow}{if} (err) \textcolor{keywordflow}{return} console.error(err)

  console.log(\textcolor{stringliteral}{'success!'})
\})

fs.removeSync(\textcolor{stringliteral}{'/home/jprichardson'}) \textcolor{comment}{//I just deleted my entire HOME directory.}
\end{DoxyCode}


\subsubsection*{write\+Json(file, object, \mbox{[}options\mbox{]}, callback)}

Writes an object to a J\+S\+O\+N file. {\ttfamily options} are the same that you\textquotesingle{}d pass to \href{https://github.com/jprichardson/node-jsonfile#writefilefilename-options-callback}{\tt `json\+File.write\+File()`}.

Alias\+: {\ttfamily write\+J\+S\+O\+N()}

Sync\+: {\ttfamily write\+Json\+Sync()}, {\ttfamily write\+J\+S\+O\+N\+Sync()}

Example\+:


\begin{DoxyCode}
var fs = require(\textcolor{stringliteral}{'fs-extra'})
fs.writeJson('./package.json', \{name: \textcolor{stringliteral}{'fs-extra'}\}, \textcolor{keyword}{function} (err) \{
  console.log(err)
\})
\end{DoxyCode}


\subsection*{Third Party }

\subsubsection*{Promises}

Use \href{https://github.com/petkaantonov/bluebird}{\tt Bluebird}. See \href{https://github.com/petkaantonov/bluebird/blob/master/API.md#promisification}{\tt https\+://github.\+com/petkaantonov/bluebird/blob/master/\+A\+P\+I.\+md\#promisification}. {\ttfamily fs-\/extra} is explicitly listed as supported.


\begin{DoxyCode}
var Promise = require(\textcolor{stringliteral}{'bluebird'})
var fs = Promise.promisifyAll(require('fs-extra'))
\end{DoxyCode}


Or you can use the package \href{https://github.com/overlookmotel/fs-extra-promise}{\tt `fs-\/extra-\/promise`} that marries the two together.

\subsubsection*{Type\+Script}

If you like Type\+Script, you can use {\ttfamily fs-\/extra} with it\+: \href{https://github.com/borisyankov/DefinitelyTyped/tree/master/fs-extra}{\tt https\+://github.\+com/borisyankov/\+Definitely\+Typed/tree/master/fs-\/extra}

\subsubsection*{File / Directory Watching}

If you want to watch for changes to files or directories, then you should use \href{https://github.com/paulmillr/chokidar}{\tt chokidar}.

\subsubsection*{Misc.}


\begin{DoxyItemize}
\item \href{https://github.com/cadorn/mfs}{\tt mfs} -\/ Monitor your fs-\/extra calls.
\end{DoxyItemize}

\subsection*{Hacking on fs-\/extra }

Wanna hack on {\ttfamily fs-\/extra}? Great! Your help is needed! \href{http://nodei.co/npm/fs-extra.png?downloads=true&downloadRank=true&stars=true}{\tt fs-\/extra is one of the most depended upon Node.\+js packages}. This project uses \href{https://github.com/feross/standard}{\tt Java\+Script Standard Style} -\/ if the name or style choices bother you, you\textquotesingle{}re gonna have to get over it \+:) If {\ttfamily standard} is good enough for {\ttfamily npm}, it\textquotesingle{}s good enough for {\ttfamily fs-\/extra}.

\href{https://github.com/feross/standard}{\tt !\mbox{[}js-\/standard-\/style\mbox{]}(https\+://cdn.\+rawgit.\+com/feross/standard/master/badge.\+svg)}

What\textquotesingle{}s needed?
\begin{DoxyItemize}
\item First, take a look at existing issues. Those are probably going to be where the priority lies.
\item More tests for edge cases. Specifically on different platforms. There can never be enough tests.
\item Really really help with the Windows tests. See appveyor outputs for more info.
\item Improve test coverage. See coveralls output for more info.
\item A directory walker. Probably this one\+: \href{https://github.com/thlorenz/readdirp}{\tt https\+://github.\+com/thlorenz/readdirp} imported into {\ttfamily fs-\/extra}.
\item After the directory walker is integrated, any function that needs to traverse directories like {\ttfamily copy}, {\ttfamily remove}, or {\ttfamily mkdirs} should be built on top of it.
\item After the aforementioned functions are built on the directory walker, {\ttfamily fs-\/extra} should then explicitly support wildcards.
\end{DoxyItemize}

Note\+: If you make any big changes, {\bfseries you should definitely post an issue for discussion first.}

\subsection*{Naming }

I put a lot of thought into the naming of these functions. Inspired by \textquotesingle{}s request. So he deserves much of the credit for raising the issue. See discussion(s) here\+:


\begin{DoxyItemize}
\item \href{https://github.com/jprichardson/node-fs-extra/issues/2}{\tt https\+://github.\+com/jprichardson/node-\/fs-\/extra/issues/2}
\item \href{https://github.com/flatiron/utile/issues/11}{\tt https\+://github.\+com/flatiron/utile/issues/11}
\item \href{https://github.com/ryanmcgrath/wrench-js/issues/29}{\tt https\+://github.\+com/ryanmcgrath/wrench-\/js/issues/29}
\item \href{https://github.com/substack/node-mkdirp/issues/17}{\tt https\+://github.\+com/substack/node-\/mkdirp/issues/17}
\end{DoxyItemize}

First, I believe that in as many cases as possible, the \href{http://nodejs.org/api/fs.html}{\tt Node.\+js naming schemes} should be chosen. However, there are problems with the Node.\+js own naming schemes.

For example, {\ttfamily fs.\+read\+File()} and {\ttfamily fs.\+readdir()}\+: the {\bfseries F} is capitalized in {\itshape File} and the {\bfseries d} is not capitalized in {\itshape dir}. Perhaps a bit pedantic, but they should still be consistent. Also, Node.\+js has chosen a lot of P\+O\+S\+I\+X naming schemes, which I believe is great. See\+: {\ttfamily fs.\+mkdir()}, {\ttfamily fs.\+rmdir()}, {\ttfamily fs.\+chown()}, etc.

We have a dilemma though. How do you consistently name methods that perform the following P\+O\+S\+I\+X commands\+: {\ttfamily cp}, {\ttfamily cp -\/r}, {\ttfamily mkdir -\/p}, and {\ttfamily rm -\/rf}?

My perspective\+: when in doubt, err on the side of simplicity. A directory is just a hierarchical grouping of directories and files. Consider that for a moment. So when you want to copy it or remove it, in most cases you\textquotesingle{}ll want to copy or remove all of its contents. When you want to create a directory, if the directory that it\textquotesingle{}s suppose to be contained in does not exist, then in most cases you\textquotesingle{}ll want to create that too.

So, if you want to remove a file or a directory regardless of whether it has contents, just call {\ttfamily fs.\+remove(path)} or its alias {\ttfamily fs.\+delete(path)}. If you want to copy a file or a directory whether it has contents, just call {\ttfamily fs.\+copy(source, destination)}. If you want to create a directory regardless of whether its parent directories exist, just call {\ttfamily fs.\+mkdirs(path)} or {\ttfamily fs.\+mkdirp(path)}.

\subsection*{Credit }

{\ttfamily fs-\/extra} wouldn\textquotesingle{}t be possible without using the modules from the following authors\+:


\begin{DoxyItemize}
\item \href{https://github.com/isaacs}{\tt Isaac Shlueter}
\item \href{https://github.com/avianflu}{\tt Charlie Mc\+Connel}
\item \href{https://github.com/substack}{\tt James Halliday}
\item \href{https://github.com/andrewrk}{\tt Andrew Kelley}
\end{DoxyItemize}

\subsection*{License }

Licensed under M\+I\+T

Copyright (c) 2011-\/2015 \href{https://github.com/jprichardson}{\tt J\+P Richardson} 