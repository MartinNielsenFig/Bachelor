\href{http://travis-ci.org/gotwarlost/istanbul}{\tt !\mbox{[}Build Status\mbox{]}(https\+://secure.\+travis-\/ci.\+org/gotwarlost/istanbul.\+png)} \href{https://gemnasium.com/gotwarlost/istanbul}{\tt !\mbox{[}Dependency Status\mbox{]}(https\+://gemnasium.\+com/gotwarlost/istanbul.\+png)} \href{https://coveralls.io/r/gotwarlost/istanbul?branch=master}{\tt !\mbox{[}Coverage Status\mbox{]}(https\+://img.\+shields.\+io/coveralls/gotwarlost/istanbul.\+svg)} \href{https://www.bithound.io/github/gotwarlost/istanbul}{\tt !\mbox{[}bit\+Hound Score\mbox{]}(https\+://www.\+bithound.\+io/github/gotwarlost/istanbul/badges/score.\+svg)}

\href{https://nodei.co/npm/istanbul/}{\tt !\mbox{[}N\+P\+M\mbox{]}(https\+://nodei.\+co/npm/istanbul.\+png?downloads=true)}


\begin{DoxyItemize}
\item \href{#features}{\tt Features and use cases}
\item \href{#getting-started}{\tt Getting started and configuration}
\item \href{#the-command-line}{\tt The command line}
\item \href{#ignoring-code-for-coverage}{\tt Ignoring code for coverage}
\item \href{#api}{\tt A\+P\+I}
\item https\+://github.com/gotwarlost/istanbul/blob/master/\+C\+H\+A\+N\+G\+E\+L\+O\+G.\+md \char`\"{}\+Changelog\char`\"{}
\item \href{#license}{\tt License and credits}
\end{DoxyItemize}

\subsubsection*{Features}


\begin{DoxyItemize}
\item All-\/javascript instrumentation library that tracks {\bfseries statement, branch, and function coverage}.
\item {\bfseries Module loader hooks} to instrument code on the fly
\item {\bfseries Command line tools} to run node unit tests \char`\"{}with coverage turned on\char`\"{} and no cooperation whatsoever from the test runner
\item Multiple report formats\+: {\bfseries H\+T\+M\+L}, {\bfseries L\+C\+O\+V}, {\bfseries Cobertura} and more.
\item Ability to use as \href{https://github.com/gotwarlost/istanbul-middleware}{\tt middleware} when serving J\+S files that need to be tested on the browser.
\item Can be used on the {\bfseries command line} as well as a {\bfseries library}
\item Based on the awesome {\ttfamily esprima} parser and the equally awesome {\ttfamily escodegen} code generator
\item Well-\/tested on node (prev, current and next versions) and the browser (instrumentation library only)
\end{DoxyItemize}

\subsubsection*{Use cases}

Supports the following use cases and more


\begin{DoxyItemize}
\item transparent coverage of nodejs unit tests
\item instrumentation/ reporting of files in batch mode for browser tests
\item Server side code coverage for nodejs by embedding it as \href{https://github.com/gotwarlost/istanbul-middleware}{\tt custom middleware}
\end{DoxyItemize}

\subsubsection*{Getting started}

\begin{DoxyVerb}$ npm install -g istanbul
\end{DoxyVerb}


The best way to see it in action is to run node unit tests. Say you have a test script {\ttfamily test.\+js} that runs all tests for your node project without coverage.

Simply\+: \begin{DoxyVerb}$ cd /path/to/your/source/root
$ istanbul cover test.js
\end{DoxyVerb}


and this should produce a {\ttfamily coverage.\+json}, {\ttfamily lcov.\+info} and {\ttfamily lcov-\/report/$\ast$html} under {\ttfamily ./coverage}

Sample of code coverage reports produced by this tool (for this tool!)\+:

\href{http://gotwarlost.github.com/istanbul/public/coverage/lcov-report/index.html}{\tt H\+T\+M\+L reports}

\subsubsection*{Configuring}

Drop a {\ttfamily .istanbul.\+yml} file at the top of the source tree to configure istanbul. {\ttfamily istanbul help config} tells you more about the config file format.

\subsubsection*{The command line}

\begin{DoxyVerb}$ istanbul help
\end{DoxyVerb}


gives you detailed help on all commands.


\begin{DoxyCode}
1 Usage: istanbul help config | <command>
2 
3 `config` provides help with istanbul configuration
4 
5 Available commands are:
6 
7       check-coverage
8               checks overall/per-file coverage against thresholds from coverage
9               JSON files. Exits 1 if thresholds are not met, 0 otherwise
10 
11 
12       cover   transparently adds coverage information to a node command. Saves
13               coverage.json and reports at the end of execution
14 
15 
16       help    shows help
17 
18 
19       instrument
20               instruments a file or a directory tree and writes the
21               instrumented code to the desired output location
22 
23 
24       report  writes reports for coverage JSON objects produced in a previous
25               run
26 
27 
28       test    cover a node command only when npm\_config\_coverage is set. Use in
29               an `npm test` script for conditional coverage
30 
31 
32 Command names can be abbreviated as long as the abbreviation is unambiguous
\end{DoxyCode}


To get detailed help for a command and what command-\/line options it supports, run\+: \begin{DoxyVerb}istanbul help <command>
\end{DoxyVerb}


(Most of the command line options are not covered in this document.)

\paragraph*{The {\ttfamily cover} command}

\begin{DoxyVerb}$ istanbul cover my-test-script.js -- my test args
# note the -- between the command name and the arguments to be passed
\end{DoxyVerb}


The {\ttfamily cover} command can be used to get a coverage object and reports for any arbitrary node script. By default, coverage information is written under {\ttfamily ./coverage} -\/ this can be changed using command-\/line options.

The {\ttfamily cover} command can also be passed an optional {\ttfamily -\/-\/handle-\/sigint} flag to enable writing reports when a user triggers a manual S\+I\+G\+I\+N\+T of the process that is being covered. This can be useful when you are generating coverage for a long lived process.

\paragraph*{The {\ttfamily test} command}

The {\ttfamily test} command has almost the same behavior as the {\ttfamily cover} command, except that it skips coverage unless the {\ttfamily npm\+\_\+config\+\_\+coverage} environment variable is set.

{\bfseries This command is deprecated} since the latest versions of npm do not seem to set the {\ttfamily npm\+\_\+config\+\_\+coverage} variable.

\paragraph*{The {\ttfamily instrument} command}

Instruments a single J\+S file or an entire directory tree and produces an output directory tree with instrumented code. This should not be required for running node unit tests but is useful for tests to be run on the browser.

\paragraph*{The {\ttfamily report} command}

Writes reports using {\ttfamily coverage$\ast$.json} files as the source of coverage information. Reports are available in multiple formats and can be individually configured using the istanbul config file. See {\ttfamily istanbul help report} for more details.

\paragraph*{The {\ttfamily check-\/coverage} command}

Checks the coverage of statements, functions, branches, and lines against the provided thresholds. Positive thresholds are taken to be the minimum percentage required and negative numbers are taken to be the number of uncovered entities allowed.

\subsubsection*{Ignoring code for coverage}


\begin{DoxyItemize}
\item Skip an {\ttfamily if} or {\ttfamily else} path with {\ttfamily /$\ast$ istanbul ignore if $\ast$/} or {\ttfamily /$\ast$ istanbul ignore else $\ast$/} respectively.
\item For all other cases, skip the next \textquotesingle{}thing\textquotesingle{} in the source with\+: {\ttfamily /$\ast$ istanbul ignore next $\ast$/}
\end{DoxyItemize}

See ignoring-\/code-\/for-\/coverage.md for the spec.

\subsubsection*{A\+P\+I}

All the features of istanbul can be accessed as a library.

\paragraph*{Instrument code}


\begin{DoxyCode}
1 var istanbul = require('istanbul');
2 var instrumenter = new istanbul.Instrumenter();
3 
4 var generatedCode = instrumenter.instrumentSync('function meaningOfLife() \{ return 42; \}',
5     'filename.js');
\end{DoxyCode}


\paragraph*{Generate reports given a bunch of coverage J\+S\+O\+N objects}


\begin{DoxyCode}
1 var istanbul = require('istanbul'),
2     collector = new istanbul.Collector(),
3     reporter = new istanbul.Reporter(),
4     sync = false;
5 
6 collector.add(obj1);
7 collector.add(obj2); //etc.
8 
9 reporter.add('text');
10 reporter.addAll([ 'lcov', 'clover' ]);
11 reporter.write(collector, sync, function () \{
12     console.log('All reports generated');
13 \});
\end{DoxyCode}


For the gory details consult the \href{http://gotwarlost.github.com/istanbul/public/apidocs/index.html}{\tt public A\+P\+I}

\subsubsection*{Multiple Process Usage}

Istanbul can be used in a multiple process environment by running each process with Istanbul, writing a unique coverage file for each process, and combining the results when generating reports. The method used to perform this will depend on the process forking A\+P\+I used. For example when using the \href{http://nodejs.org/api/cluster.html}{\tt cluster module} you must setup the master to start child processes with Istanbul coverage, disable reporting, and output coverage files that include the P\+I\+D in the filename. Before each run you may need to clear out the coverage data directory.


\begin{DoxyCode}
1 if(cluster.isMaster) \{
2     // setup cluster if running with istanbul coverage
3     if(process.env.running\_under\_istanbul) \{
4         // use coverage for forked process
5         // disabled reporting and output for child process
6         // enable pid in child process coverage filename
7         cluster.setupMaster(\{
8             exec: './node\_modules/.bin/istanbul',
9             args: [
10                 'cover', '--report', 'none', '--print', 'none', '--include-pid',
11                 process.argv[1], '--'].concat(process.argv.slice(2))
12         \});
13     \}
14     // ...
15     // ... cluster.fork();
16     // ...
17 \} else \{
18     // ... worker code
19 \}
\end{DoxyCode}


\subsubsection*{Coverage.\+json}

For details on the format of the coverage.\+json object, ./coverage.json.\+md \char`\"{}see here\char`\"{}.

\subsubsection*{License}

istanbul is licensed under the \href{http://github.com/gotwarlost/istanbul/raw/master/LICENSE}{\tt B\+S\+D License}.

\subsubsection*{Third-\/party libraries}

The following third-\/party libraries are used by this module\+:


\begin{DoxyItemize}
\item abbrev\+: \href{https://github.com/isaacs/abbrev-js}{\tt https\+://github.\+com/isaacs/abbrev-\/js} -\/ to handle command abbreviations
\item async\+: \href{https://github.com/caolan/async}{\tt https\+://github.\+com/caolan/async} -\/ for parallel instrumentation of files
\item escodegen\+: \href{https://github.com/Constellation/escodegen}{\tt https\+://github.\+com/\+Constellation/escodegen} -\/ for J\+S code generation
\item esprima\+: \href{https://github.com/ariya/esprima}{\tt https\+://github.\+com/ariya/esprima} -\/ for J\+S parsing
\item fileset\+: \href{https://github.com/mklabs/node-fileset}{\tt https\+://github.\+com/mklabs/node-\/fileset} -\/ for loading and matching path expressions
\item handlebars\+: \href{https://github.com/wycats/handlebars.js/}{\tt https\+://github.\+com/wycats/handlebars.\+js/} -\/ for report template expansion
\item js-\/yaml\+: \href{https://github.com/nodeca/js-yaml}{\tt https\+://github.\+com/nodeca/js-\/yaml} -\/ for Y\+A\+M\+L config file load
\item mkdirp\+: \href{https://github.com/substack/node-mkdirp}{\tt https\+://github.\+com/substack/node-\/mkdirp} -\/ to create output directories
\item nodeunit\+: \href{https://github.com/caolan/nodeunit}{\tt https\+://github.\+com/caolan/nodeunit} -\/ dev dependency for unit tests
\item nopt\+: \href{https://github.com/isaacs/nopt}{\tt https\+://github.\+com/isaacs/nopt} -\/ for option parsing
\item once\+: \href{https://github.com/isaacs/once}{\tt https\+://github.\+com/isaacs/once} -\/ to ensure callbacks are called once
\item resolve\+: \href{https://github.com/substack/node-resolve}{\tt https\+://github.\+com/substack/node-\/resolve} -\/ for resolving a post-\/require hook module name into its main file.
\item rimraf -\/ \href{https://github.com/isaacs/rimraf}{\tt https\+://github.\+com/isaacs/rimraf} -\/ dev dependency for unit tests
\item which\+: \href{https://github.com/isaacs/node-which}{\tt https\+://github.\+com/isaacs/node-\/which} -\/ to resolve a node command to a file for the {\ttfamily cover} command
\item wordwrap\+: \href{https://github.com/substack/node-wordwrap}{\tt https\+://github.\+com/substack/node-\/wordwrap} -\/ for prettier help
\item prettify\+: \href{http://code.google.com/p/google-code-prettify/}{\tt http\+://code.\+google.\+com/p/google-\/code-\/prettify/} -\/ for syntax colored H\+T\+M\+L reports. Files checked in under {\ttfamily lib/vendor/}
\end{DoxyItemize}

\subsubsection*{Inspired by}


\begin{DoxyItemize}
\item Y\+U\+I test coverage -\/ \href{https://github.com/yui/yuitest}{\tt https\+://github.\+com/yui/yuitest} -\/ the grand-\/daddy of J\+S coverage tools. Istanbul has been specifically designed to offer an alternative to this library with an easy migration path.
\item cover\+: \href{https://github.com/itay/node-cover}{\tt https\+://github.\+com/itay/node-\/cover} -\/ the inspiration for the {\ttfamily cover} command, modeled after the {\ttfamily run} command in that tool. The coverage methodology used by istanbul is quite different, however
\end{DoxyItemize}

\subsubsection*{Shout out to}


\begin{DoxyItemize}
\item \href{https://github.com/mfncooper}{\tt mfncooper} -\/ for great brainstorming discussions
\item \href{https://github.com/reid}{\tt reid}, \href{https://github.com/davglass}{\tt davglass}, the Y\+U\+I dudes, for interesting conversations, encouragement, support and gentle pressure to get it done \+:)
\end{DoxyItemize}

\subsubsection*{Why the funky name?}

Since all the good ones are taken. Comes from the loose association of ideas across coverage, carpet-\/area coverage, the country that makes good carpets and so on... 